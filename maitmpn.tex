\documentclass[11pt,a4paper,polish,thesis]{dcsbook}
\usepackage[utf8]{inputenc}
\usepackage[polish]{babel}
%\usepackage[T1]{fontenc}
\usepackage{import}
\usepackage{tabularx}
\usepackage{subfig}
\usepackage{listings}
\usepackage{float}
\usepackage{marginnote}
\usepackage{multirow}
\usepackage{ltablex}
\usepackage{booktabs}
\usepackage{graphicx}

\graphicspath{ {images/} }

% nie dzielić!
% \hyphenation{Qwer Tyui}
\sloppy

\setcounter{secnumdepth}{4}
\setcounter{tocdepth}{2}
\begin{document}
\author{Michał Buszkiewicz, Bartosz Kostaniak}
\title{Wielowersyjny model spójności replik dla systemu gromadzenia danych w modelu klucz-wartość}
\supervisor{dr inż.~Dariusz Wawrzyniak}
\date{Poznań, 2017}
\maketitle
\frontmatter
\tableofcontents{}
\mainmatter

% 1 Wprowadzenie
\import{chapters/}{01-intro.tex}

% 2 Problematyka replikacji danych
\import{chapters/}{02-replication.tex}

% 3 Przegląd istniejących rozwiązań
\import{chapters/}{03-existingsolutions.tex}

% 4 Opis proponowanego systemu
\import{chapters/}{04-systemdescription.tex}

% 5 Implementacja
\import{chapters/}{05-implementation.tex}

% 6 Testy wydajnościowe
% ===== poszły do 5 ======
% \import{chapters/}{06-perftests.tex}
% Na sieci lokalnej i rozległej, np. porównanie wydajności dla poszczególnych spójności. Być może warto byłoby użyć jakiegoś bardziej izolowanego środowiska (DW wspomniał o jakimś HPC gdzie można byłoby postawić to bez wirtualizacji), ale zostaje problem symulowania opóźnień sieci rozległych.

% 6 Podsumowanie i wnioski
\import{chapters/}{06-summary.tex}

\backmatter

% Testing bibtex
\bibliography{chapters/bibliography}
\bibliographystyle{ieeetr}

\end{document}
