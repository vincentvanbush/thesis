\chapter{Wstęp} \label{chapter:intro}

Upowszechnienie oraz wzrost możliwości automatycznego przetwarzania danych przez systemy komputerowe zrodziły potrzebę ujmowania, gromadzenia i udostępniania informacji w formie cyfrowej, tym samym przyczyniły się do rozwoju technologii składowania danych. Składowanie danych stało się ostatecznie jedną z podstawowych funkcjonalności systemów informatycznych, przesuwając ich przetwarzania niejako na dalszy plan. Znamienny jest tu fakt, że jednostki zajmujące się przetwarzaniem danych masowych, nazywane od lat \textit{centrami obliczeniowymi}, określane są współcześnie jako \textit{centra danych}.

Ciągły wzrost potrzeb w zakresie dostępu do informacji wymaga poszukiwania zarówno bardziej efektywnych sposobów zapisu danych (tak, by zmieścić ich jak najwięcej w możliwie małej przestrzeni), jak i szybszej ich transmisji zarówno do podzespołów odpowiedzialnych za ich przetwarzanie, jak i pomiędzy poszczególnymi fizycznymi nośnikami (tak, by zagwarantować odpowiedni czas dostępu do rosnącej objętości danych). Technologie składowania danych podlegają zatem nieustannej ewolucji, będącej wypadkową takich czynników, jak: rozwój fizycznych nośników trwałego przechowywania danych, na przykład poprzez zwiększanie gęstości zapisu, oraz rozbudowa technik transmisji danych, obejmująca między innymi dążenie do zwiększania prędkości transferu czy też zmniejszania opóźnień.

Istotny wkład w funkcjonalność technologii składowania danych wniosły rozproszone systemy informatyczne, których dynamiczny rozwój jest ściśle powiązany z powstawaniem w latach 70-tych i 80-tych XX wieku rezległych sieci komputerowych. Rozwijane w ostatnich dekadach XX wieku technologie baz danych stanowiły podstawę do badania problematyki udostępniania danych w systemach rozproszonych --- z myślą o przystosowaniu zbiorów danych do dostępu zdalnego w sieciach charakteryzujących się znacznymi odległościami między węzłami i słabymi gwarancjami dotyczącymi ich niezawodności. System rozproszony powinien przy tym zapewniać pewien rodzaj transparentności, rozumianej najczęściej jako nierozróżnialność takiego podejścia ze względu na jego zachowanie od rozwiązania operatego na zasobach lokalnych.

Wzrost przepustowości łączy komunikacyjnych pozwala na coraz szersze stosowanie rozproszonych systemów gromadzenia i udostępniania informacji w sytuacjach, w których używane są urządzenia o ograniczonych możliwościach lokalnego składowania danych. Tego typu rozwiązania stanowią najczęściej nadbudowę (tzw.\ \textit{middleware}) nad istniejącymi już systemami składowania lokalnego, kontrolując utrzymywanie replik na wielu węzłach i spójność danych.

% Niniejsza praca stanowi próbę zaprojektowania i implementacji systemu, który --- opierając się o
% istniejące rozwiązania składowania lokalnego, funkcjonując jako warstwa tzw.\ \textit{middleware} --- zapewnia warunki do składowania danych z szerokimi możliwościami kontroli jego działania przez twórcę aplikacji konsumującej dane.

\section{Cele i zakres pracy}

Głównym celem niniejszej pracy dyplomowej jest opracowanie i prototypowa implementacja mechanizmu replikacji dla systemu bazy danych typu NoSQL w modelu klucz-wartość, który umożliwia dostęp do danych zgodnie z modelem spójności wyspecyfikowanym przy zlecaniu operacji, wyrażonym jako kombinacja modelu danocentrycznego i/lub wymagań odnośnie gwarancji sesji.

Jako wypadkowa analizy istniejących rozwiązań oraz opisywanych w literaturze modeli spójności i innych uwarunkowań funkcjonujących w systemach replikowanych, przedstawiona zostanie koncepcja algorytmu koordynacji replik stanowiąca podstawę do prototypowej implementacji w języku Scala, a także do dalszych rozważań dotyczących możliwych usprawnień systemu. Powstałe rozwiązanie zostanie przeanalizowane pod kątem efektywnościowym od strony teoretycznej oraz w praktyce.

% \section{Zakres pracy}
Wykonana praca obejmuje następujące elementy:

\begin{itemize}
    \item Przegląd rozwiązań w zakresie systemów baz danych typu klucz-wartość oraz mechanizmów replikacji w tych systemach ze szczególnym uwzględnieniem problematyki spójności.
    % [BK] Nie przypominam sobie, aby w pracy były jakieś "mechanizmy synchronizacji danych"
    % [MB] Takie pojęcie pojawiało się w karcie pracy i uważam że spokojnie możemy rozumieć to jako metody, których używają istniejące systemy do koordynacji replik danych.
    \item Przegląd modeli spójności replik oraz mechanizmów synchronizacji danych.
    \item Opracowanie koncepcji udostępniania replik w wielowersyjnym modelu spójności.
    \item Realizacja opracowanej koncepcji w prototypowym systemie bazy danych typu klucz-wartość.
    \item Analiza możliwości realizacji wielowersyjnego modelu spójności w istniejących systemach typu NoSQL.

\end{itemize}

\section{Podział zadań}
Zadania składające się na wykonaną pracę są podzielone w następujący sposób:
\begin{itemize}
    \item Michał Buszkiewicz: przegląd rozwiązań w zakresie systemów baz danych typu klucz wartość oraz mechanizmów replikacji w tych systemach, opracowanie koncepcji udostępniania danych w wielowersyjnym modelu spójności, implementacja interfejsu dostępu do danych, testy poprawności.
    \item Bartosz Kostaniak: przegląd modeli spójności replik oraz mechanizmów synchronizacji danych, opracowanie koncepcji synchronizacji replik, implementacja protokołu spójności, testy wydajnosci.
\end{itemize}


\section{Struktura pracy}

% 3 Przegląd istniejących rozwiązań
% Redis, memcached, systemy stricte do replikowania danych

% 4 Opis proponowanego systemu
% Koncepcja, pseudokody, rysunki, schematy objaśniające, własności z punktu widzenia istotnych kryteriów

% 5 Implementacja
% Szczegóły - język, biblioteki, struktura projektu, klas ...

% 6 Testy wydajnościowe
% Na sieci lokalnej i rozległej, np.\ porównanie wydajności dla poszczególnych spójności. Być może warto byłoby użyć jakiegoś bardziej izolowanego środowiska (DW wspomniał o jakimś HPC gdzie można byłoby postawić to bez wirtualizacji), ale zostaje problem symulowania opóźnień sieci rozległych.

% 7 Podsumowanie i wnioski

Materiał, jaki obejmuje niniejsza praca dyplomowa, został przedstawiony w następującym porządku:
\begin{itemize}

\item Rozdział \ref{chapter:replication} (,,Problematyka replikacji danych'') zawiera przegląd zagadnień związanych ze zwielokrotnianiem kopii danych i problematyką spójności w ujęciu danocentrycznym, jak również klientocentrycznym --- poprzedzony zdefiniowaniem modelu systemu rozproszonego jako podstawy do dalszych rozważań.

\item Rozdział \ref{chapter:existingsolutions} (,,Przegląd istniejących rozwiązań'') opisuje funkcjonujące na rynku systemy baz danych typu klucz-wartość ze szczególnym uwzględnieniem rozwiązań replikacji danych, co stanowi punkt odniesienia dla określenia cech, jakimi ma charakteryzować się wytworzona w ramach pracy aplikacja.

\item Rozdział \ref{chapter:systemdescription} (,,Opis proponowanego systemu'') nakreśla wypracowaną w ramach niniejszej pracy koncepcję systemu poprzez zaprezentowanie pseudokodów i schematów z objaśnieniem działania opracowanych algorytmów, a także analizą z punktu widzenia istotnych kryteriów poprawności, spójności i złożoności komunikacyjnej.

% [BK] można dać jakieś zdanie o tolerancji awarii - to też jest w tym rozdziale
% [MB] :+1:
\item Rozdział \ref{chapter:implementation} (,,Implementacja'') prezentuje implementacyjne aspekty wykonanego prototypu opartego na opisanej koncepcji --- język programowania, zastosowane biblioteki, techniczną strukturę projektu. Rozdział ten ustosunkowuje się także do problematyki tolerancji awarii, wskazując braki systemu w tym zakresie oraz propozycje ich rozwiązań.

\item Rozdział \ref{chapter:summary} (,,Podsumowanie'') zamyka pracę dyplomową bilansem wykonanych prac i rozważaniem na temat kierunków możliwych dalszych prac nad systemem.

\end{itemize}

\subsection*{Zawartość płyty CD}

Do niniejszej pracy dyplomowej została dołączona płyta CD zawierająca pracę dyplomową w formatach PDF i LaTeX oraz kod źródłowy prototypowego systemu wytworzonego jako element pracy.
