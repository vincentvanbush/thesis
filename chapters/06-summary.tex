\chapter{Podsumowanie} \label{chapter:summary}

W ramach niniejszej pracy magisterskiej zrealizowano rozproszony system gromadzenia danych typu klucz-wartość, realizujący koncepcję wielowersyjności danych. Koncepcja ta rozumiana jest jako możliwość zlecania systemowi operacji dostępu do danych (odczyt lub zapis) wraz z wyspecyfikowaniem wymagań dotyczących danocentrycznego modelu spójności, który ma zostać zachowany przy dostępie do danej i/lub wymagań dotyczących gwarancji sesji (modeli spójności klientocentrycznych). Żądania dostępu do danych przy zachowaniu słabszych modeli mogą być zrealizowane w ogólnym przypadku szybciej, a wygenerowana przez system odpowiedź zawiera świeższe wartości danych, choć mniej można wówczas zakładać na temat ich spójności. Analogicznie, żądając danych zgodnych z silnymi modelami spójności opóźniamy odpowiedź od systemu lub otrzymana odpowiedź będzie zawierała starsze wersje danych. To właśnie stworzenie systemu zachowującego opisywaną powyżej elastyczność w dostępie do danych było główną ideą przyświecającą autorom w trakcie projektowania i implementowania proponowanego rozwiązania. 

Udało się zaprojektować oraz zaimplementować rozproszony system składowania danych typu klucz-wartość, który nie narzuca użytkownikom jednego poziomu spójności danych, z którego byliby zmuszeni korzystać, ale pozwala na elastyczne dobieranie gwarancji, z którymi mają być dostępne dane. Pozwala to na optymalizację działania aplikacji poprzez minimalizację czasu oczekiwania na dane lub korzystanie z możliwie najświeższych wersji danych. Dodatkowo, dopóki użytkownik nie korzysta z gwarancji sesji, czas odpowiedzi systemu na żądania jest bardzo krótki niezależnie od wybranego modelu danocentrycznego, w zgodzie z którym należy wykonać operację, ponieważ zapisy będą wówczas nieblokujące, a odczyt zwróci dane tym starsze im silniejszy wybrano model spójności. Należy w tym miejscu podkreślić, że jest to podejście rzadko spotykane w istniejących rozwiązaniach, co potencjalnie może stanowić przyczynek do dalszych prac nad gotowymi do pracy w środowiskach produkcyjnych systemami opartymi w swojej konstrukcji na podobnych założeniach.

Projekt architektury i towarzysząca mu prototypowa implementacja systemu posiadają pewne niedoskonałości i braki, do których warto byłoby odnieść się w dalszych pracach. System w obecnym kształcie nie zapewnia odporności na podziały sieci. W przypadku istnienia rozłącznych partycji, obrazy bazy danych w poszczególnych częściach sieci będą wraz z kolejnymi zapisami coraz bardziej różniły się od siebie, co stanowi poważny problem, ponieważ nie przewidziano mechanizmu wykrywania takich awarii ani reakcji w sytuacji ponownego połączenia się wielu partycji w jedną całość. Ponadto, problemem systemu jest możliwość wystąpienia awarii wyróżnionego procesu sekwencera. W takim przypadku postęp przetwarzania, tj. materializowanie kolejnych wartości staje się niemożliwe. Niniejsza praca zawiera jedynie wskazówki co do kierunków działań w celu rozwiązania tych problemów, jednak od strony implementacyjnej nie zostały one wprowadzone w życie.

Proponowany system nie wspiera także możliwości podziału przechowywanych danych pomiędzy węzły (ang. \textit{sharding}), co mogłoby być pożądane przez użytkowników dla lepszego skalowania wraz ze wzrostem ilości przechowywanych danych. Obecna implementacja nie wyklucza jednak możliwości dodania takiej funkcjonalności w przyszłości. W przypadku, gdyby praca miałaby być dalej rozwijana, stanowi to dobry punkt wyjścia do kontynuacji badań.

Podsumowując, implementacja proponowanego rozwiązania spełniła podstawowe, składane przed nią wymagania, umożliwiając swobodne dobieranie wymagań dotyczących danocentrycznego modelu spójności i/lub gwarancji sesji, które mają zostać zachowane przy wykonywaniu operacji dostępu do danych. Cecha ta stanowi wyróżnik systemu na tle istniejących rozwiązań, ograniczających lub w ogóle eliminujących możliwość doboru poziomu spójności do każdego wykonywanego żądania, lub definiujących poziomy spójności według innych kryteriów, nie odnosząc się do modeli dano- i klientocentrycznych. Koncepcja wielowersyjności okazała się więc możliwa do zrealizowania w praktyce, a jej wykorzystanie do przechowywania danych pozwoliłoby na efektywniejszy dostęp do nich. Mimo że opisywany w niniejszej pracy system nie stanowi rozwiązania gotowego do wdrożenia w środowisku produkcyjnym, to jest pierwszym krokiem do stworzenia takiej bazy danych, czy to poprzez rozbudowę, czy to poprzez wykorzystanie wiedzy i rozwiązań zawartych w niniejszym projekcie.