\chapter{Wstęp} \label{chapter:intro}

Programowanie gier komputerowych jest dynamicznie rozwijającym się polem zastosowania nauk informatycznych, korzystającym zarazem z dorobku wielu innych dziedzin nauki - w szczególności nauk fizycznych i matematycznych. Jednym z wiodących zagadnień w tym obszarze jest tematyka zastosowania w tworzonych grach mechanizmów sztucznej inteligencji (ang. \textit{Artificial Intelligence, AI}), które mają na celu symulowanie zachowań postaci i zjawisk występujących w grze tak, by odzwierciedlać w świecie gry zachowania istot inteligentnych.

Zagadnienie wykorzystania sztucznej inteligencji w grach jest poruszane w programach dydaktycznych wielu uczelni technicznych --- w tym Politechniki Poznańskiej --- jako część modułów dotyczących programowania gier. Realizowany na Wydziale Informatyki Politechniki Poznańskiej proces dydaktyczny przedmiotu ,,Programowanie gier'' obejmuje naukę implementacji mechanizmów AI za pomocą interfejsu skryptowego zintegrowanego z grą. Zadania tego interfejsu są dwojakie --- pierwsze z nich to przekazanie danych o otaczającym świecie w którym toczy się akcja gry, natomiast drugie to zbiór podstawowych akcji, które postać może wykonać.

% , udostępniającego podstawowe akcje, jakie może wykonywać postać w grze, a także dane o jej percepcji otaczającego świata, na podstawie których podejmuje decyzje.


\section{Sztuczna inteligencja w grach komputerowych}

Z purystycznego punktu widzenia mechanizmy AI stosowane w grach komputerowych nie spełniają w pełni definicji sztucznej inteligencji - nie realizują bowiem w rzeczywistości celu, jaki przyświeca tej dziedzinie w jej akademickim ujęciu, czyli emulacji zachowań istot inteligentnych. W praktyce, gra komputerowa - w zależności od konkretnego przypadku - realizuje jedynie niektóre jej cele poprzez pewne reguły, często mające charakter przybliżony. Cel stosowania AI ogranicza się w tym przypadku do zapewnienia graczowi pożądanego zakresu wrażeń podczas rozgrywki \cite{adaptiveai}. Niemniej jednak zwiększające się możliwości obliczeniowe komputerów pozwalają na inkorporowanie coraz większej liczby elementów sztucznej inteligencji (w ścisłym jej rozumieniu) do powstających gier komputerowych.

\subsection*{Rys historyczny}
Wykorzystanie sztucznej inteligencji w grach komputerowych sięga pierwszych prób ich implementacji. W 1951 roku w Nowym Jorku opracowano skomputeryzowaną wersję chińskiej łamigłówki matematycznej \textit{Nim}, będącą w stanie regularnie pokonywać nawet doświadczonych graczy. W tym samym roku na uniwersytecie w Manchesterze powstały programy grające w szachy oraz warcaby. \iffalse{odniesienie do: http://www.newyorker.com/magazine/1952/08/02/it}\fi 

Lata 70-te XX wieku przyniosły powstanie pierwszych gier wideo, przeznaczonych dla automatów lub komputerów osobistych, w tym także tytułów umożliwiających grę z udziałem przeciwników sterowanych przez mechanizmy AI. Wśród nich były gry implementujące prosty ruch przeciwników po planszy, a także symulatory wyścigów samochodowych i gry typu \textit{shooter}.

Rozwój mikroprocesorów na przełomie lat 70. i 80. umożliwił udoskonalanie stosowanych mechanizmów AI, modelujących zachowania sterowanych postaci za pomocą zdarzeń zależnych od liczb pseudolosowych i funkcji haszujących (\textit{Space Invaders, Pac-Man}). Sztuczna inteligencja była stopniowo wprowadzana do bijatyk i fabularnych gier akcji (\textit{Role Playing Game, RPG}), a także do gier sportowych, np. jako mechanizm umożliwiający wcielenie się komputera w rolę menedżera drużyny.

Od lat 90-tych stosowano coraz bardziej wyrafinowane metody implementacji sztucznej inteligencji w grach, takie jak: automaty stanowe, drzewa decyzyjne, logika zbiorów rozmytych czy sieci neuronowe. Oprócz prześcigania się w tworzeniu coraz bardziej wyrafinowanej warstwy graficznej --- tradycyjnego pola rywalizacji producentów gier komputerowych --- autorzy gier podjęli pogoń za jak najlepszymi wrażeniami z samej rozgrywki, czego esencją jest rozwój mechanizmów AI \cite{adaptiveai}.

\subsection*{Wykorzystanie systemów skryptowych}

W tak złożonych grach jak choćby komputerowe gry fabularne (cRPG), czy też strategiczne, w których liczba możliwych wyborów przy każdym posunięciu osiąga liczby rzędu setek lub nawet tysięcy, kluczową rolę w uproszczeniu implementacji AI odgrywają systemy skryptowe. Dzieje się tak ze względu na ich czytelność, przewidywalność, możliwość adaptacji do okoliczności, a także fakt, iż mogą być używane przez osoby niebędące programistami \cite{adaptiveai}. Języki skryptowe, jako interpretowane i niewymagające kompilacji, pozwalają na sprawniejszy rozwój elementów gier ulegających częstym modyfikacjom --- takich jak właśnie sztuczna inteligencja.

Języki skryptowe w grach mają różnorodne zastosowania --- zarówno do tworzenia dodatków do już kompletnego oprogramowania (np. World of Warcraft \cite{wowlua}), jak również jako integralna część procesu wytwarzania gry, ułatwiająca projektantom definiować znaczne części mechaniki gry \cite{oreilly}.

\section{Cele i zakres pracy}

Głównym celem niniejszej pracy dyplomowej jest utworzenie platformy do nauki pisania skryptów AI dla gier komputerowych opartej na nowoczesnym środowisku Unreal Engine~4. Przeznaczeniem dla niej jest wykorzystanie podczas zajęć dydaktycznych na Wydziale Informatyki Politechniki Poznańskiej. Jednocześnie ma pozostać zachowana zgodność pod względem interfejsu skryptowego z dotychczas stosowanymi rozwiązaniami.
% dopisać że to pochodzi od evaLUAtion

% \section{Zakres pracy}
Wykonana praca obejmowała następujące elementy:

\begin{itemize}
    \item	Zapoznanie się ze środowiskiem Unreal Engine 4.
    \item	Zapoznanie się ze skryptami Lua i platformą evaLUAtion.
    \item	Refaktoryzacja platformy evaLUAtion i jej integracja z silnikiem Unreal Engine.
    \item	Implementacja edytora map z zachowaniem wstecznej kompatybilności.
    \item	Przygotowanie komponentów środowiska UE umożliwiających przeprowadzenie turnieju  skryptów walczących.

\end{itemize}

\section{Podział zadań}
Wśród elementów zakresu pracy, wszyscy członkowie zespołu zapoznali się wstępie ze środowiskiem Unreal Engine 4, skryptami Lua i platformą evaLUAtion. Pozostałe elementy zakresu pracy zostały zdekomponowane na mniejsze zadania i wykonane przez następujące osoby:
\begin{itemize}
    \item Michał Buszkiewicz -- interfejs użytkownika aplikacji, edytor map, integracja interpretera języka skryptowego Lua i sterowania postaciami przez skrypty; opracowanie rozdziałów \ref{chapter:intro} i \ref{chapter:evaluation2},
    \item Wojciech Grzeszczak -- opracowanie funkcjonalności rozgrywki z udziałem użytkownika, implementacja sterowania przy pomocą myszy i klawiatury, testowanie napisanych skryptów, system generowania połączeń pomiędzy punktami nawigacyjnymi; opracowanie rozdziałów \ref{chapter:impl} i \ref{chapter:assumptions},
    \item Mateusz Moskwa -- przygotowanie obsługi plików danych i konfiguracji, obsługa ładowania i zapisywania plików map w edytorze map i grze, wczytywanie konfiguracji rozgrywki; opracowanie rozdziałów \ref{chapter:techbasics} oraz dodatku \ref{chapter:appendixA},
    \item Krzesimir Samborski -- implementacja systemu akcji postaci oraz mechanizmów działania broni, efekty wizualne --- zjawiska cząsteczkowe i fizyczne, równoważenie efektywności broni; opracowanie rozdziałów \ref{chapter:impl} i \ref{chapter:summary}.
    % \item \textbf{Szczerze mówiąc, nie bardzo mamy jeszcze pojęcie, jak ugryźć ten punkt.}
    % \item Michał Buszkiewicz -- 
    % \item Wojciech Grzeszczak -- system nawigacji, strzelanie fizycznymi(?) pociskami, trzecia wersja systemu akcji(częściowo, jak to zaznaczyć?), naprawa błędów oraz poprawki, usprawnienia i parę innych rzeczy, wczytywanie map w edytorze, snap to grid, snap to walls, kamera TPP, celownik w trybie TPP, LATARKI, eksplodujące ściany, %no wojtek nie słodź już sobie :P :DDDD
    % \item Mateusz Moskwa -- 
    % \item Krzesimir Samborski -- oprawa audiowizualna, system akcji
\end{itemize}


\section{Struktura pracy}

Materiał, jaki obejmuje niniejsza praca dyplomowa, został przedstawiony w następującym porządku:
\begin{itemize}
\item Rozdział \ref{chapter:techbasics} (,,Przykładowe technologie wytwarzania gier komputerowych'') przedstawia przykładowe technologie, języki programowania, środowiska i narzędzia programistyczne stosowane w tworzeniu gier.

\item Rozdział \ref{chapter:assumptions} (,,Założenia'') stanowi opis głównych założeń i wymagań projektu, z uwzględnieniem istniejącej platformy.

\item Rozdział \ref{chapter:evaluation2} (,,Platforma evaLUAtion2'') jest z jednej strony opisem wytworzonego systemu, a z drugiej podręcznikiem użytkownika powstałej aplikacji.

\item Rozdział \ref{chapter:impl} (,,Implementacja'') prezentuje najistotniejsze informacje dotyczące implementacji projektu i rozwiązań napotykanych w tym czasie problemów technicznych i organizacyjnych.

\item Rozdział \ref{chapter:summary}} (,,Podsumowanie'') stanowi podsumowanie pracy z uwzględnieniem przeglądu realizacji założeń oraz wniosków końcowych.

\item Dodatek \ref{chapter:appendixA} (,,Podręcznik tworzenia skryptów'') zawiera podręcznik tworzenia skryptów sztucznej inteligencji w języku Lua dla opracowanej platformy.
\end{itemize}

\subsection*{Zawartość płyty CD}

Do niniejszej pracy dyplomowej została dołączona płyta CD zawierająca wykonaną aplikację w wersji źródłowej (wraz z instrukcją budowania i niezbędnymi bibliotekami) i binarnej dla 64-bitowych systemów operacyjnych Windows. Płyta zawiera także wykaz elementów aplikacji niestanowiących części tej pracy wraz z odpowiednimi informacjami licencyjnymi. Wreszcie, na płycie znajduje się wersja elektroniczna tej pracy dyplomowej w wersjach PDF i LaTeX oraz prezentacja multimedialna na temat utworzonego projektu w formacie Microsoft PowerPoint.

% 2

% Niniejszy rozdział zawiera podsumowanie istniejących technologii, języków programowania, środowisk i narzędzi programistycznych stanowiących punkt wyjścia do realizacji niniejszej pracy inżynierskiej. Omówione zostanie środowisko Unreal Engine 4 i zastosowane w nim języki programowania Blueprint i C++, a także język skryptowy Lua zastosowany do pisania skryptów sztucznej inteligencji.

% 3

% W rozdziale tym zdefiniowane zostały główne założenia i wymagania dotyczące projektu --- z uwzględnieniem jako punktu odniesienia istniejącej już aplikacji evaLUAtion. Opisano aspekty dotyczące: oprawy wizualnej aplikacji, przebiegu rozgrywki, trybów gry, mechaniki, akcji i elementów świata gry, jakie powinny istnieć w utworzonym programie. Opisane zostały także założenia dotyczące możliwości konfiguracji rozgrywki oraz map, na jakich odbywa się gra. Na końcu rozdziału wyszczególniono wymagania dotyczące interfejsu skryptowego, do którego dostęp ma użytkownik piszący skrypty sztucznej inteligencji, a także założenia pozaimplementacyjne dotyczące wytwarzania oprogramowania.

% 4

% Niniejszy rozdział zawiera opis wykonanego systemu z punktu widzenia jego obsługi przez użytkownika. Przedstawione zostaną wszystkie dostępne w aplikacji funkcje --- narzędzia służące do przygotowania środowiska rozgrywki (edytor wykorzystywanych w grze map i menedżer profili konfiguracyjnych), opis uruchamiania rozgrywki i wyświetlanych w jej trakcie informacji, a także sterowania postaciami w trybie gry z udziałem użytkownika.

% 5

% W niniejszym rozdziale zaprezentowany zostanie sposób implementacji projektu evaLUAtion2 --- w tym schematyczna struktura systemu i ogólna zasada jego działania, a także opis napotkanych problemów wraz z ich rozwiązaniami i ich porównaniem z poprzednią wersją evaLUAtion. Rozdział objaśnia też sposób integracji interpretera języka Lua z aplikacją.

% 6

% W niniejszym rozdziale, podsumowującym niniejszą pracę dyplomową, zawarty został finalny przegląd rezultatów realizacji projektu evaLUAtion2. Przede wszystkim, podsumowano najważniejsze cechy charakteryzujące powstały system. Nakreślone zostały również perspektywy dalszego rozwijania funkcjonalności platformy --- zarówno w kontekście samej gry, jak i wartości dydaktycznej programu. Na koniec zreasumowano wnioski, do jakich dotarł zespół realizujący pracę w przebiegu implementowania projektu.